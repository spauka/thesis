% $Log: abstract.tex,v $

%% The text of your abstract and nothing else (other than comments) goes here.
%% It will be single-spaced and the rest of the text that is supposed to go on
%% the abstract page will be generated by the abstractpage environment.  This
%% file should be \input (not \include 'd) from cover.tex.
Throughout this Ph.D., the quest to build a quantum computer has accelerated, driven by ever-improving fidelities
of conventional qubits and the development of new technologies that promise topologically protected qubits with
the potential for lifetimes that exceed those of comparable conventional qubits. As such, there has been an
explosion of interest in the design of an architecture for a quantum computer. This design would have to include
high-quality qubits at the bottom of the stack, be extensible, and allow the layout of many qubits with scalable
methods for readout and control of the entire device. Furthermore, the whole experimental infrastructure must
handle the requirements for parallel operation of many qubits in the system. Hence the crux of this thesis: to
design an architecture for a semiconductor-based quantum computer that encompasses all the elements that would
be required to build a large scale quantum machine, and investigate the individual these elements at each layer
of this stack, from qubit to readout to control.

Each chapter of this thesis investigates a different layer of the stack from the top down.
In Chapter~\ref{sec:arch}, I explore the key elements of the architecture of a quantum computer.
I first define a common structure by which we can compare the control and readout hardware for
any given design, following which I present two potential architectures for a quantum computer:
one that multiplexes both readout and control from room-temperature to \SI{4}{\kelvin} in
Sec.~\ref{sec:primelines}, followed by one which uses {CryoCMOS} to generate control pulses near the qubits
in Sec.~\ref{sec:gooseberry}.

In Chapter~\ref{sec:hall}, I move up the stack and present a set of experiments concerning the design and
implementation of circulators based on the quantum Hall effect (Sec.~\ref{sec:hallcirc}) and the
anomalous quantum Hall effect (Sec.~\ref{sec:spinhallcirc}). Circulators are vital components in qubit
experiments, used to route signals and isolate qubits from thermal photons, however, are currently bulky,
centimeter-scale devices. By capacitively coupling to the edge magnetoplasmon modes of micron-scale Hall droplets,
I demonstrate non-reciprocal transmission with isolation similar to that of off-the-shelf components.

In Chapter~\ref{sec:spinqubit}, I explore the design of the spin qubit and examine the dispersive gate sensing technique,
which holds promise as a scalable method of readout for large arrays of quantum dots. First, in Sec.~\ref{sec:5dot},
the scalable design for a singlet-triplet qubit is proposed, which allows for the layout of larger arrays of qubits
while presenting a method for two-qubit coupling over intermediate length scales via an intermediate quantum state.
Then, in Sec.~\ref{sec:pockets}, I inspect anomalous signals present in dispersive gate sensors, proposing that they
are caused by localized pockets of charge that form in the 2-dimensional electron gas (2DEG). These pockets of charge may
contribute to the charge-noise that plagues semiconductor based qubits.

Finally, in Chapter~\ref{sec:majoinas}, I study the formation of Majorana zero modes in InAs nanowires and
2DEGs and the experimental challenges of realizing a qubit based on topological qubits. In Sec.~\ref{sec:inas_hb},
I propose techniques for improving the quality of shallow InAs 2DEG's after processing, which remains a limiting
factor in the design of qubits based on Majorana zero modes in InAs. In Sec.~\ref{sec:rfmajo}, I investigate
techniques for finding and reading out Majorana zero modes and present benchmarks on the current generation
of charge sensors in nanowires.