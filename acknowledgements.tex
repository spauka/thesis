\bgroup
\let\addcontentsline=\nocontentsline
\section*{Acknowledgments}
\egroup
\addtocontents{toc}{\protect\setcounter{tocdepth}{-1}}
\addcontentsline{toc}{chapter}{Acknowledgements}
\addtocontents{toc}{\protect\setcounter{tocdepth}{2}}

It was more than six years ago now that I found myself choosing between a research career in physics and
one in computer science. At the time, it was a difficult choice, but with the benefit of hindsight, I have
no doubts that I made the right call. That is to a considerable degree due to amazing people I've had the
opportunity to work with.

First, I would like to thank my supervisor David Reilly. Without his gently cajoling at the beginning, I may have ended
up somewhere entirely different. Since then, I've learned an enormous amount through his teaching and through
constant and entertaining discussions. Without his support and guidance, I can't imagine having had the
fantastic Ph.D. I have had. More than that, under his supervision, I've met extraordinary people, traveled
to countless conferences and seen the lab grow from a single fridge and a smattering of students, to an empire
with seven fridges (a number which I think is becoming eight as I write this) and an innumerable number
of people. I still remember our road trip down California and all the discussions it provoked as one of the best times I've had.
I am also grateful to Andrew Doherty who gave his time and counsel, both in physics (of which his
knowledge seems endless) and in life, despite his enormously busy schedule. Under his tutelage, I learned so much about
physics and the community, understanding to which I still regularly refer. To Maja Cassidy, who joined only recently
but has had such a massive impact on the science that I (and indeed the whole lab) has been able to do, thank you
for your support and direction, I gained so much insight in our conversations over the past year.

To the entire Quantum Nanoscience Lab, I want to express my heartfelt gratitude. The people who I've had a chance
to work with have all enriched my life, professional and personal, enormously. I can't think of a better group of
people to have worked with over the years.

To John Hornibrook, I think it was only a few months ago that I learned that our pairing in the early days of Honours
was a last minute arrangement. Thank you for patiently explaining the fundamentals of physics to a hopelessly lost
student. Without your encouragement in the early days, I can't imagine how I would have made it this far. The lab
has been a pleasure to work in by your side, and I hope we can continue to swap programming oddities well into the future.

To James Colless, who is both an unending fountain of knowledge and who taught me how to survive in a physics lab,
thank you for giving me such a great intuitive understanding of the work we do. I enjoyed working and learning with
you immensely, and to top it off, our time traveling was the best!

Xanthe Croot is just about the most joyous, tireless, optimistic and loud person I know, who brings so much to any
workplace, all the while being one of the best physicists I know. Thank you for the constant arguments(/discussions).
Even though I think we freaked out anyone walking past, they were amongst the most informative times I had.
Thanks also for pushing me to do more outside of the lab, whether it was running a half marathon or singing Mariah Carey
loudly in the lab, I grew a great deal as a result.

Last amongst the original quantum dot crew, to Alice Mahoney, thank you for being at first a fantastic mentor and then
a great friend. I still remember you stressing about my honors talk and thesis far more than me, which, you know, fair,
and as a result pushing me to do better and be better. You made the lab, and everyone in it better organized, more diligently
logbooked and pushed me to be meticulous with the science. The brunches were fun too.

To Ewa Rej, thanks for being a fantastic labmate and friend. Our trivia Wednesdays are a social event I've never quite managed
to repeat, and our conversations have meant so much over the years. MC Jarratt, thanks for being an excellent lab buddy, drinking
partner, and friend. Our road trips through America were, without a doubt, a highlight of my Ph.D., and I can't wait to go on another
craft brewery crawl. To Steven Waddy, your endless knowledge about the minutiae of physics and electronics has taught me so much and made the lab a surprisingly insightful place to work. To Ian Conway-Lamb, your help with PCB, FPGA, and mechanical engineering
was invaluable over the years, our movie nights never quite recovered after you left. To my coffee friend over in NMR: Tom Boele,
thanks for the discussions and support particularly in the last few weeks of the Ph.D. The chance to escape writing has been
indispensable for my sanity. To Torsten Gaebel, thanks for all your help fixing things, for keeping us all safe and for the
company on our many nights to the pub.

To everyone else in the QNL, thank you. The ASIC team of Kushal Das, Ali Moini and YuanYuan Yang, learning about transistors,
CMOS design, and working with you on Mulberry and Gooseberry has been invaluable. The FPGA and rf team of Jon Knoblauch, Deshan Govender
and Neil Dick, thanks for the insights about FPGA design, rf engineering, and whiskey. And remaining engineers and scientists,
Jana Darulova, Brendan Altus, Tim Newman, and Andrew Kelly, thanks for all the conversations over the years. I can't imagine a workplace
this good, and it's all down to the fantastic people in this lab.

To all the others who have supported me scientifically thank you. To the entire team at ANFF UNSW, Joanna Syzmanska, Nadia Court,
and Pierrette Michaux, thank you for your help with fabrication and for training me in the early days. To all the growers, Hong Lu and Art Gossard
from UCSB, and Geoff Gardner, John Watson and Mike Manfra from Purdue University, who have supplied us with high-quality material without
which we would not have been able to do any science, thank you. To the entire team at Copenhagen who kindly hosted me over a period of
3 months, it was an amazingly productive time, and everyone there made me feel like I had a true second home. In particular, I would like to
thank Charlie Marcus for welcoming me as if I was a longtime member of his lab, and Dovydas Razmadze, Filip Malinowski and Hung Nguyen,
who shared their time and experiments.

Oh, don't think I've forgotten you, Alexis George! From your time in Physics admin to now you've sorted out so many issues for me, even
when it really wasn't your job. Thanks for making every day I was writing enjoyable. And indeed to the rest of the School of Physics
admin and workshop, thank you for all your support over the years.

To my friends, in particular, Karl Bromfield, Matt Saddington, Adam Seage, and Harriet Rosman who supported me through these long years,
and seldom showed any doubt that I would finish, I wouldn't have been able to do it without your support. Last, to my family,
thank you for all your help over these past five years, the late night dinners, the long chats, the distractions and perspectives on life outside
academia.
