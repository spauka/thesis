\chapter{Quantum (Anomalous) Hall Effect Circulators}
\label{sec:hall}

While discussing architectural concerns in Chapter~\ref{sec:arch}, we focused a great deal on the need to reduce the complexity of interconnects,
particlarly as 2D architectures are scaled up. In doing so we discussed the need for attenuation in the control wiring of a quantum computer in order
to reduce the thermal population of photons at the qubit interface. In a similar way, there is a need for isolation in the readout chain, however
as we explored in Sec.~\ref{sec:readout}, the effective noise temperature $T_\textrm{eff}$ is set by the first stage amplifier and by any attenuation
which occurs between it and the qubit. As such, particularly for transmon-type experiments where any noise is known to limit qubit
lifetimes~\cite{PhysRevLett.101.080502}, passive circulators with low loss must be used. These are non-reciprocal devices which route power
in a single direction around their ports, i.e. from $1 \to 2$, $2 \to 3$ and $3 \to 1$ but not in the inverse direction $1 \not\to 3$, $3 \not\to 2$
and $2 \not\to 1$. In addition when using quantum limited parametric amplifiers such as the travelling-wave parametric amplifier~\cite{PhysRevLett.113.157001},
isolators are necessary to limit the action of the pump tone on our qubits.

Circulators have traditionally been constructed using the Faraday effect in magnetic materials, which causes propagation to occur at different phase
velocities in opposite directions through a magnetic material, however as these devices operate based on interference of counter-propagating
paths, these devices must be constructed with a size on the order of the wavelength of the signal, often several tens of centimeters. It was proposed by
Giovanni Viola and David DiVincenzo that slow, chiral charge density waves in the quantum Hall regine, called edge magnetoplasmons (EMPs), might be utilized to
construct millimeter-scale circulators~\cite{PhysRevX.4.021019}. In the following chapter, I present work that implements this idea to form non-reciprocal devices,
and demonstrate strong circulation (up to \SI{40}{\decibel}). The physics of EMP's is introduced originally in Sec.~\ref{sec:emp}. In Sec.~\ref{sec:hallcirc} the
absorption of EMPs is explored in GaAs, and we use these EMPs to form a micron-scale circulator. In Sec.~\ref{sec:spinhallcirc}, we use the quantum anomalous Hall
effect to create circulators that are able to operate at zero external field. The dissipation of power is investigated as a function of temperature and applied signal
amplitude, although the nature of this dissipation remains an open question.

\clearpage
\section{On-chip Microwave Quantum Hall Circulator}
\label{sec:hallcirc}
\import{chap3/}{qhe_circ}

\clearpage
\section{Zero-Field Edge Plasmons in a Magnetic Topological Insulator}
\label{sec:spinhallcirc}
\import{chap3/}{ti_circ}
