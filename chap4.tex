\chapter{Spin Qubits and Readout}
\label{sec:spinqubit}

Up to this point, the results of my thesis have been focused on the architecture and instrumentation that surrounds a quantum
computer. We switch tack at this point to focus on the bottom of the quantum computing stack; on the design of qubits, in
particular the design of spin qubits in GaAs. Although, as we discussed in Sec.~\ref{sec:dotqubits}, there have been many
successful attempts at forming spin qubits, one of the fundamental challenges facing all of the qubit designs in semiconductors
is achieving strong two-qubit interactions. In the short range, direct exchange and capacative coupling have generated two qubit
fidelities that exceed 98\%~\cite{PhysRevA.99.042310}, however as we discussed in Chapter \ref{sec:arch}, a scaleable qubit architecture
will likely require both long-range and intermediate-range couplers in order to be feasible. The designs of progressively
larger grids of quantum dots poses an additional challenge for the initialization of qubits located in the central regions
of an array. Typically, initialization of a qubit is performed by loading a known spin state from proximal reservoirs~\cite{petta}, however
for large arrays of quantum dots, such an approach is infeasible. Several recent papers suggest progressive loading of electrons
from the center out~\cite{PhysRevApplied.6.054013,qubyte}, however such schemes require a complete reset of the entire set of qubits.
The first part of this chapter, Sec.~\ref{sec:5dot}, explores an architecture that addresses the issue of initialization and of
intermediate-range coupling between singlet-triplet qubits.

On top of the coupling challenge, conventional charge-based readout techniques for spin qubits require bulky, proximal charge sensors, an
approach which carries inherent challenges for a 2D qubit layout. Alternative approaches to readout based on dispersive gate
sensing~\cite{PhysRevLett.110.046805}, while having now demonstrated single-shot readout~\cite{Nnano_dzurak}, still achieve only 73\% readout
fidelity, well below the 99.86\% level achieved using charge sensing~\cite{Keith_2019,PhysRevX.8.021046}. Understanding the limits of gate-based
readout, including understanding the sources of anomalous error signals, will be crucial to utilizing gate-based sensors for scaleable quantum
computers. This is the key problem we explore in the second part of this chapter, Sec.~\ref{sec:pockets}. I will also mention that the results
of that chapter point towards a potential source of charge noise in semiconductor-based qubits, although the degree to which they might
be eliminated by induced-electron device structures remains an open question.

\clearpage
\section{Device Architecture for Coupling Spin Qubits Via an Intermediate Quantum State}
\label{sec:5dot}
\import{chap4/}{5dot}

\clearpage
\section{Gate-Sensing Charge Pockets in the Semiconductor-Qubit Environment}
\label{sec:pockets}
\import{chap4/}{pockets}