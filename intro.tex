\chapter{Introduction}

The invention of quantum mechanics early in the twentieth century created the most complete and accurate
theory of reality that has been discovered so far. It was Richard Feynmann who theorized in his 
seminal 1981 keynote\cite{Feynman1982} that with quantum physics we could build a quantum simulator --- 
a machine that would be able to solve a class of problem that we simply couldn't solve with a
classical computer (an idea we will expand on in~\ref{sec:qc}). Since he delivered this keynote, the
field of quantum computing has exploded. First came theoretical demonstrations of algorithms
with a quantum speedup; Deutsch's Algorithm in 1992~\cite{Deutsch} and Shor's Algorithm in 1994. 
Despite these advances, many suspected that it was only a matter of time before a "no-go" result would
be found; a result that would say that quantum computers simply could not scale, or that errors in a 
quantum system would be uncorrectable.
However, with the formulation of the quantum fault-tolerance theorem~\cite{1996quant.ph.11025A}, which
showed that for a small error rate it is possible to correct errors faster than they occur, the last
reasonable objection to quantum computing was overcome (my favourite reference as to why this seems true
is in chapter 14 of Scott Aaronson's book~\cite{Aaronson:skepticism}). Since then, a plethora of physical
% TODO: add references to implementations here
systems have emerged that seem like contenders for building a quantum computer, such as trapped 
ions, nuclear spins, electron spins in semiconductors, excitations in superconductors, single photons
or a large number of other systems that are too numerous to list here. Each of them aims to realize a qubit,
the quantum equivalent of a bit, which rather than being described as a single number taking the value of
0 or 1, is described by a two-dimensional vector that evolves under the rules of quantum physics. Today,
many of these systems are realizing larger and larger systems and are butting up against the fault tolerance 
threshold, raising the specter of large quantum computers in the near future. 

The rapid progress made in the field, while no doubt exciting, also highlights a difficulty I have in
preparing this thesis. Between when I started my PhD in 2014 and now, the community underwent a seismic
shift in ambition, moving from trying to work on one or few qubit systems~\cite{iarpa_mqco} to trying 
to implement useful machines with hundreds of qubits~\cite{Monroe440}, with a concomitant increase in
funding. Industrial players have also
entered the ring trying to build viable commercial quantum machines, including IBM, Intel, Google, Rigetti, 
DWave and Microsoft. Over the same period, our lab grew from one with a single dilution refridgerator (DR) 
and four other PhD students, to one with 7 DRs, 2 cryostats, close to 50 people and substantial backing
from industry (Microsoft). It is in that context that this thesis is written. All the topics presented have the
same aim: to build a useful quantum computer, but experiments span from low level materials challenges
for building qubits, to designing individual qubits, to scalable instrumentation design, and finally architecture
design for building large scale quantum machines. 

I've grouped results into these four broad sections, each of which present a number of papers dealing 
with these results, with the aim of creating a coherent storyline around my work, starting from the top
level architecture and working my way down to materials science. The structure of this
thesis then is as follows:

\subsubsection{Chapter 1}
General Outline

\subsubsection{Chapter 2}
Architecture

\subsubsection{Chapter 3}
Quantum Hall Circulators

\subsubsection{Chapter 4}
Spin qubits and Readout

\subsubsection{Chapter 5}
InAs

\subsubsection{Chapter 6}
Conclusion

