\chapter{Introduction}

The invention of quantum mechanics early in the twentieth century created the most complete and accurate
theory of reality that has been discovered so far. It was Richard Feynman who theorized in his
seminal 1981 keynote \cite{Feynman1982} that with quantum physics we could build a quantum simulator ---
a machine that would be able to solve a class of problem that we couldn't solve with a
classical computer (an idea we will expand on in Section~\ref{sec:qc}). Since he delivered this keynote, the
field of quantum computing has exploded. First came theoretical demonstrations of algorithms
with a quantum speedup; Deutsch's Algorithm in 1992 \cite{Deutsch} and Shor's Algorithm in 1994 \cite{Shor}.
Despite these advances, many suspected that it was only a matter of time before a "no-go" result would
be found; a result that would say that quantum computers could not scale, or that errors in a quantum system would be uncorrectable.
However, with the formulation of the quantum fault-tolerance theorem \cite{1996quant.ph.11025A,doi:10.1098/rspa.1998.0167}, which
showed that for a small error rate it is possible to correct errors faster than they occur, the last
reasonable objection to quantum computing was overcome (my favourite reference as to why this seems true
is in Chapter 14 of Scott Aaronson's book \cite{Aaronson:skepticism}). Since then, a plethora of physical
systems have emerged that seem like contenders for building a quantum computer, such as trapped
ions~\cite{doi:10.1063/1.5088164}, nuclear spins~\cite{acs.nanolett.8b00006}, electron spins in
semiconductors~\cite{RevModPhys.79.1217}, excitations in superconductors~\cite{Wendin_2017},
single photons~\cite{OBrien1567}, or a large number of other systems that are too numerous to list here.
Each of them aims to realize a qubit, the quantum equivalent of a bit, which rather than being described
as a single number taking the value of 0 or 1, is represented by a two-dimensional vector that evolves under the rules
of quantum physics. Today, many of these qubits are being realized in larger and larger numbers with error rates
that are butting up against the fault tolerance threshold, raising the specter of large quantum computers
in the near future.

The rapid progress made in the field, while no doubt exciting, also highlights the difficulty I have in
preparing this thesis. Between when I started my Ph.D. in 2014 and now, the community underwent a seismic
shift in ambition, moving from trying to work on one or few-qubit systems \cite{iarpa_mqco} to trying
to implement useful machines with hundreds of qubits \cite{Monroe440}, with a concomitant increase in
funding. Industrial players have also
entered the ring trying to build viable commercial quantum machines, including IBM, Intel, Google, Rigetti,
DWave and Microsoft. Over the same period, our lab grew from one with a single dilution refrigerator (DR) and
four other Ph.D. students to one with 7 DRs, 2 cryostats, close to 50 people and substantial backing
from industry (Microsoft). It is in that context that this thesis is written. All the topics presented have the
same aim: to build a useful quantum computer, but experiments span from exploring low-level materials challenges,
to designing individual qubits, to scalable instrumentation design, and finally to architecture
designs for building large scale quantum machines.

I have grouped results into these four broad sections, each of which presents several papers dealing
with these results, intending to create a coherent storyline around my work, starting from the top
level architecture and working my way down to materials science. The structure of this thesis then is as follows:

\medskip
\noindent\textbf{\hyperref[sec:quest]{Chapter 1 - The Quest for a Quantum Computer}}

\noindent
Chapter 1 provides a brief introduction to the key concepts and background that is required to understand
the experiments that are presented in this thesis. We will start off by motivating the quest to build a quantum
computer and explore the theory that underlies the experiments. We will then look at building qubits,
the fundamental building block of a quantum computer in various semiconductor systems, as well as taking a
deeper dive into the materials systems that allows them to work. This includes a description of the band structure
of III-V semiconductors, the 2-dimensional electron gas, scattering, and the spin-orbit interaction. We discuss
the quantum Hall effect and the anomalous quantum Hall effect, and talk about how each element is brought together
to form Majorana zero modes in superconductor/semiconductor hybrids.

\newpage
\noindent\textbf{\hyperref[sec:arch]{Chapter 2 - Architecture of a Quantum Computer}}

\noindent
Chapter 2 explores how the various elements described throughout this thesis might be combined to form
a large-scale quantum computer. Firstly, the aims for the field of quantum computing in the short and medium term are
enumerated. A general framework for discussing architecture is presented, and, briefly, the problems surrounding control
and readout are elucidated. Following this, two architectures are presented that solve various aspects of the control
and readout challenge. First, in Section~\ref{sec:primelines}, the prime-lines architecture explores how high bandwidth
wiring requirements for control may be reduced. Second, in Section~\ref{sec:gooseberry}, a CryoCMOS architecture for fast
pulsing and control is described.

\medskip
\noindent\textbf{\hyperref[sec:hall]{Chapter 3 - Quantum (Anomalous) Hall Effect Circulators}}

\noindent
Chapter 3 describes the use of edge magnetoplasmons in the quantum Hall effect and the quantum anomalous Hall effect
for passive signal routing in quantum experiments. First, we discuss the challenge of isolation and signal routing
for quantum computing applications. Second, in Section~\ref{sec:hallcirc}, we explore the microwave properties of a
Hall droplet in \ce{GaAs}/\ce{(Al,Ga)As} in a magnetic field coupled to a transmission line, and then use these chiral
excitations to form a circulator. Finally, in Section~\ref{sec:spinhallcirc}, we investigate a circulator built in a topological
insulator (TI) \ce{Cr-(Bi,Sb)_2Te3}, which allows magnetic-field free operation. Dissipation mechanisms in this 3D-TI
are explored.

\medskip
\noindent\textbf{\hyperref[sec:spinqubit]{Chapter 4 - Spin Qubits and Readout}}

\noindent
Chapter 4 describes experiments on spin qubits formed in a \ce{GaAs}/\ce{(Al,Ga)As} heterostructure. By this point, we've moved
on from high-level architecture and component development to qubit development, and although these experiments focus on
spin qubits, many of the issues and solutions raised have applicability to any dispersive readout or quantum dot system.
First, in Section~\ref{sec:5dot}, we scrutinize a scaleable device design for coupled singlet-triplet qubits. Using an
extended quantum dot, we create an architecture that allows coupling over intermediate distances and addresses the challenges
of loading large arrays of quantum dots. Second, in Section~\ref{sec:pockets}, we probe anomalous sensing signals when
using dispersive gate sensing to read out a quantum dot. We associate these signals with the disordered potential landscape
of the 2DEG and explore the properties of these signals. These charge pockets are hypothesized to be a source of unexplained
charge noise in qubit devices.

\newpage
\noindent\textbf{\hyperref[sec:majoinas]{Chapter 5 - Majoranas and InAs}}

\noindent
Finally, in Chapter 5, we survey the nascent field of forming Majorana zero modes to use as topologically protected qubits. While
such a topologically protected qubit has not yet been demonstrated, if they are realized they may have lifetimes that far exceed those of
any other qubit technology. While promising, the prerequisites for forming Majorana zero modes in semiconductor/superconductor hybrid structures;
a large spin-orbit interaction, large Land\'e g-factor, high mobility, and a close, transparent interface to a superconductor, means that extensive
materials and process development must be done to enable their formation. Furthermore, existing techniques for readout and control of semiconductor
qubits must be adapted to this new architecture. In Section~\ref{sec:rfmajo}, we evaluate techniques for charge sensing of Majorana zero modes formed
in proximitized InAs nanowires. Finally, in Section~\ref{sec:inas_hb}, we investigate methods for repairing the damage done to the surface of shallow
2DEGs in InAs after processing, either using TMA to remove the dirty native oxide or an ArH plasma to passivate charged surface states, and characterize
the quality of these materials after treatment.