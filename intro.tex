\chapter{Introduction}

The invention of quantum mechanics early in the twentieth century created the most complete and accurate
theory of reality that has been discovered so far. It was Richard Feynman who theorized in his
seminal 1981 keynote \cite{Feynman1982} that with quantum physics we could build a quantum simulator ---
a machine that would be able to solve a class of problem that we couldn't solve with a
classical computer (an idea we will expand on in Section~\ref{sec:qc}). Since he delivered this keynote, the
field of quantum computing has exploded. First came theoretical descriptions of algorithms
with a quantum speedup: Deutsch's Algorithm in 1992 \cite{Deutsch} and Shor's Algorithm in 1994 \cite{Shor}.
Despite these advances, many suspected that it was only a matter of time before a "no-go" result would
be found; a result that would say that quantum computers could not scale, or that errors in a quantum system would be uncorrectable.
However, with the formulation of the quantum fault-tolerance theorem \cite{1996quant.ph.11025A,doi:10.1098/rspa.1998.0167}, which
showed that for a sufficiently small error rate it is possible to correct errors faster than they occur, the last
reasonable objection to quantum computing was overcome (my favorite reference as to why this seems true
is in Chapter 14 of Scott Aaronson's book \cite{Aaronson:skepticism}). Since then, a plethora of physical
systems have emerged that seem like contenders for building a quantum computer, such as trapped
ions~\cite{doi:10.1063/1.5088164}, nuclear spins~\cite{acs.nanolett.8b00006}, electron spins in
semiconductors~\cite{RevModPhys.79.1217}, excitations in superconductors~\cite{Wendin_2017},
single photons~\cite{OBrien1567}, or a large number of other systems that are too numerous to list here.
Each of them aims to realize a qubit, the quantum equivalent of a bit, which rather than being described
as a single number taking the value of 0 or 1, is represented by a two-dimensional vector that evolves under the rules
of quantum physics, an idea I expand upon in Section~\ref{sec:qubit}. Today, many of these qubits are being
realized in larger and larger numbers with error rates that are butting up against the fault-tolerance threshold,
raising the specter of large quantum computers in the near future.

With such a large number of physical systems to choose from, and with significant differences in the experimental
apparatus necessary to form and control them, I make the choice to limit the discussion in my thesis to semiconductor and superconductor
based qubit systems. This class of qubit still encompasses a significant portion of the above systems, however, these systems
have the advantage of sharing a number of features which will allow us to reason about the design of a large-scale quantum machine,
despite the fact that a truly scalable quantum chip is not yet a reality. Namely, these each have all electrical control and
readout with bandwidths up to a few \si{\giga\hertz}, and must all be cooled to temperatures of a few \si{\milli\kelvin} in order to protect the delicate
quantum information stored within them from thermal excitation, and for superconducting and topological variants, to form a superconducting
state. In particular, the requirement to cool these systems introduces limitations in the power that may be dissipated proximally to the quantum device,
with only a few micro-Watt's of cooling power available when cooling devices to \si{\milli\kelvin} in a cryostat. With this in mind,
the design of a quantum computer must balance power, interconnection and latency between the different stages to fit within the noise,
space and power constraints of the system.

The rapid progress made in the field, while no doubt exciting, also highlights the difficulty I have in
preparing this thesis. Between when I started my Ph.D. in 2014 and now, the community underwent a seismic
shift in ambition, moving from trying to work on one or few-qubit systems \cite{iarpa_mqco} to trying
to implement useful machines with hundreds of qubits \cite{Monroe440}, with a concomitant increase in
funding. Industrial players have also entered the ring trying to build viable commercial quantum machines,
including IBM, Intel, Google, Rigetti, DWave and Microsoft. Over the same period, our lab grew from one with a
single dilution refrigerator (DR) and four other Ph.D. students to one with 7 DRs, 2 cryostats, close to 50 people
and substantial backing from industry (Microsoft). It is in that context that this thesis is written. All the topics
presented have the same aim: to build a useful quantum computer; but experiments span from exploring low-level materials
challenges, to designing individual qubits, to scalable instrumentation design, and finally to architecture
designs for building large scale quantum machines.

I have grouped results into five broad chapters, each of which presents several papers dealing
with these results, intending to create a coherent story-line around my work, starting from the top
level architecture in Chapter~\ref{sec:arch} and working my way down to materials science in Chapter~\ref{sec:majoinas}.
A brief description of the structure of this thesis is as follows.

In order to discuss how different architecture designs for a quantum computer vary, the topic of Chapter~\ref{sec:arch},
we must first devise a system for comparing them
based on common features of their design. Existing techniques for comparing designs, based on those used in classical computing
such as Rent's rule~\cite{5388820}, have recently been adapted for their quantum counterparts~\cite{FRANKE20191}, however are
not sufficiently descriptive for describing or contrasting the different architectures that have been proposed. I will therefore begin, in
Section~\ref{sec:archdesign}, by defining a new quantum-specific framework for discussing architectures used for quantum computers,
using the three criteria (power, interconnection and latency) highlighted above. Following this, I present two new architectures,
the prime lines architecture in Section~\ref{sec:primelines} which routes signals generated at room temperature between many qubits,
and a CryoCMOS architecture which is able to generate pulses at \si{\milli\kelvin}. Each of these significantly reduces the number of interconnects necessary between
room temperature and \si{\milli\kelvin}, allowing the number of qubits controlled by finite resources to be increased dramatically. Such approaches,
while previously proposed in various theoretical forms~\cite{10.1038/npjqi.2015.11,s41467-017-01905-6}, are realized for the first
time in Section~\ref{sec:gooseberry}.

To enable the architectures proposed in this thesis, it is crucial to, in parallel, develop low- or no-power technologies that
form the building blocks of the routing and interconnection in this thesis, which may be tightly integrated with the qubit
chip at the bottom of the fridge (see Section~\ref{sec:archdesign} for the argument as to why this must be done at \si{\milli\kelvin}). This idea is explored
in Chapters~\ref{sec:arch} and~\ref{sec:hall}. The goal of tighter integration is achieved by three methods. First, utilizing the
technologies used to make qubits for routing and control, as is done with the reflective switches that enable the prime-lines
architecture in Section~\ref{sec:primelines}. Second is miniaturization of existing components to allow tight integration and
further scaling of existing designs, as is presented in the work miniaturizing circulators in Chapter~\ref{sec:hall}. Finally,
the utilization of existing technologies, in this case CMOS, for qubit control, is described in Section~\ref{sec:gooseberry}.

All the above discussion hinges on the demonstration of high-fidelity qubits, without which quantum computing will remain just
a hypothetical idea. With that in mind, the final two chapters of this thesis deal with the development of scalable
designs for high-fidelity qubits. Two ideas are explored. Chapter~\ref{sec:spinqubit} studies a well-established qubit platform: the spin
qubit. Although this platform has been studied extensively in the past~\cite{RevModPhys.75.1,RevModPhys.79.1217}, scaling
them to useful sizes requires numerous further improvements in design, control and readout. For example, existing designs
require complex protocols for initialization~\cite{PhysRevApplied.6.054013} or the placement of many large proximal charge
sensors~\cite{s41467-019-09194-x}. In this thesis, I tackle these two issues, presenting an tileable design of a spin-qubit
device in GaAs that allows simple initialization of qubits in long chains in Section~\ref{sec:5dot}, and investigate sources
of noise that limits the performance of dispersive gate sensing, a technique that allows the use of gate electrodes rather
than bulky proximal charge sensors for readout of qubits, in Section~\ref{sec:pockets}. Although the experiments in this thesis
focus on devices in a \ce{GaAs}/\ce{(Al,Ga)As} heterostructure, many of the issues and solutions raised have applicability to any
dispersive readout or quantum dot system.

Finally, in Chapter 5, we survey the nascent field of forming Majorana zero modes to use as topologically protected qubits. While
such a topologically protected qubit has not yet been demonstrated, if they are realized they may have lifetimes that far exceed those of
any other qubit technology~\cite{s41578-018-0003-1}. While promising, the prerequisites for forming Majorana zero modes in semiconductor/superconductor
hybrid structures: a large spin-orbit interaction, large Land\'e g-factor, high mobility, and a close, transparent interface to a superconductor, means that
extensive materials and process development must be performed before they are realized. Furthermore, existing techniques for readout and control of semiconductor
qubits must be adapted to this new architecture. In Section~\ref{sec:inas_hb}, we investigate methods for repairing the damage done to the surface of shallow
2DEGs in InAs after processing, either using TMA to remove the dirty native oxide or an ArH plasma to passivate charged surface states, and characterize
the quality of these materials after treatment. Then, in Section~\ref{sec:rfmajo}, we evaluate techniques for charge sensing of Majorana zero modes formed
in proximitized InAs nanowires.

Before delving into new results, let's start off with a brief review of the key concepts in quantum computing and materials science necessary
to understand the results that follow and the context in which this research is performed.